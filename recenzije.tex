

 % !TEX encoding = UTF-8 Unicode

 \documentclass[a4paper]{report}

 \usepackage[T2A]{fontenc} % enable Cyrillic fonts
 \usepackage[utf8x,utf8]{inputenc} % make weird characters work
 \usepackage[serbian]{babel}
 %\usepackage[english,serbianc]{babel}
 \usepackage{amssymb}
 
 \usepackage{color}
 \usepackage{url}
 \usepackage[unicode]{hyperref}
 \hypersetup{colorlinks,citecolor=green,filecolor=green,linkcolor=blue,urlcolor=blue}
 
 \newcommand{\odgovor}[1]{\textcolor{blue}{#1}}
 
 \begin{document}
 
 \title{Dopunite naslov svoga rada\\ \small{Dopunite autore rada}}
 
 \maketitle
 
 \tableofcontents
 

 \chapter{Recenzent \odgovor{--- ocena: 4} }
 
 
 \section{O čemu rad govori?}
 % Напишете један кратак пасус у којим ћете својим речима препричати суштину рада (и тиме показати да сте рад пажљиво прочитали и разумели). Обим од 200 до 400 карактера.
 Ovaj rad govori o kriterijumima koje je potrebno ispuniti da bi studije bile idealne. Rezultati rada izvedeni su iz ankete sprovedene nad studentima Matematičkog fakulteta. U prvom delu rada postavljeni su kriterijumi koje je autor smatrao bitnim, a zatim je izvršena anketa kako bi se utvrdilo da li su studenti saglasni sa misljenjem autora. Kao najvažniji kriterijumi izdvajaju se zadovoljstvo studenata kursevima studija - konkretno osnova koja se dobija njihovim pohađanjem i potražnja za stečenim znanjima na tršižtu, organizacija fakulteta i podrška studentima u dodatnom usavršavanje, kao i komunikacija sa profesorima i asistentima. Pored samog navođenja kriterijuma, agrumentovano je i zašto su baš oni važni i koje posledice imaju na kvalitet studija. U drugom delu rada obrađeni su rezultati ankete čiji je cilj bio utvrđivanje ispunjenosti ovih kriterijuma na Matematičkom fakultetu. Analizom odgovora došlo se do zaključka da Matematički fakultet ne ispunja uslove potrebne za idealne studije. Uočava se da je najveći problem nedostatak komunikacije izmedju studenata i profesora i nezainteresovanost same organizacije fakulteta za studentska misljenja i potrebe. Čitanjem rada stiče se utisak da fakultet dovoljno dobro priprema studente za nastavak karijera nakon studija, tj. da pruža kvalitetna i osnovna znanja, ali da ne posvećuje dovoljno pažnje svim drugim organizacionim aspektima poput obezbeđivanja materijala, praksi, načina polaganja i rasporedima aktivnosti. U pojedinim delovima predstavljeni su i predlozi kako poboljšati utvđene kriterijume kako bi se podigao nivo kvaliteta studija.
 
 
 \section{Krupne primedbe i sugestije}
 % Напишете своја запажања и конструктивне идеје шта у раду недостаје и шта би требало да се промени-измени-дода-одузме да би рад био квалитетнији.
 Iako postoji relevantna literatura, smatram da su obrađeni kriterijumi mogli biti potkrepljeni zvaničnim, odnosno globalnim istraživanjima (ili ukoliko u navedenoj literaturi postoje ovakva istraživanja, treba ih citirati u delovima rada koji se na njih oslanjaju). Rad se bazira isključivo na misljenju autora i studenata o tome šta je važno da bi studije bile idealne. Smatram da su se mogli ispitati i kriterijumi koje studenti smatraju manje važnim i koji su postavljeni nekim globalnim standardom jer studenti možda nisu uvek svesni kako bi neke ideje i prakse doprinele njihovom napretku.\\
 \odgovor{Teško je pronaći literaturu koja se bavi konkretno temom idealnih studija, zato su za potrebe ovog rada korišćeni izvori koji se bave temom konkretnog kriterijuma o kom se u tom delu teksta diskutuje. Ideja je bila da izdvojimo kriterijume koje su ispitanici smatrali bitnim, pa da njihove stavove potvrdimo pomoću drugih istraživanja koja navode kako ispunjavanje ili neispunjavanje tog kriterijuma utiče na kvalitet studija i napredak studenata.}\\
 Iako rad ima dosta stilskih propusta i možda bi predstavljanje kriterijuma moglo da se obradi detaljnije, smatram da je zanimljiv i da su autori obradili važne aspekte. Pogotovo je važna analiza za Matematički fakultet.\\
 \odgovor{Stilske greške su otklonjene i delovi teksta koji su delovali neformalno su preformulisani da bi bili formalniji.}
 
 
 \section{Sitne primedbe}
 % Напишете своја запажања на тему штампарских-стилских-језичких грешки
 Detaljnim čitanjem rada, uočeno je nekoliko štamparskih grešaka i gramatičkih propusta. Što se tiče stilskih grešaka na više mesta uočena je pogrešna upotreba reči (''svoj'' umesto ''njihov'', ''ipak'' u rečenici čiji smisao ne zahteva upotrebu). Kao glavnu manu ističem formu rečenica: pojedine rečenice su duge, delovi rečenica su nekorelisani što otežava razumevanje, dok su neke od njih pisane neformalnih govorom.
 
 
 \odgovor{Štamparske, stilske  i gramatičke greške su otklonjene.}
 \odgovor{U uvodu preformulisan tekst da bude formalniji. Izbačeno korišćenje prvog lica množine (npr. smatramo) i izbačena rečenica \emph{Da je lako, do sad bi verovatno sve studije bile idealne} jer ne doprinosi nikako, a neformalno je. Urađene i male izmene kako bi uvod bio jasniji i lakši za čitanje. Promenjen sažetak, prva rečenica skroz promenjena, malo prepravljene ostale rečenice kako bi bile jasnije, formalnije i kraće. U podnaslovu 4.4 izbaceno koriscenje licnih zamenica u mnozini zbog formalnosti. Zakljucak preformulisan. }
 
 \section{Provera sadržajnosti i forme seminarskog rada}
 % Oдговорите на следећа питања --- уз сваки одговор дати и образложење
 
 \begin{enumerate}
 \item Da li rad dobro odgovara na zadatu temu?\\
 Smatram da rad dobro odgovara na zadatu temu. Odgovara na postavljena pitanja iz zahteva, a pored toga sadrži i dodatna.
 
 \item Da li je nešto važno propušteno?\\
 U okviru sekcije \textit{Krupne primedbe i sugestije} navela sam primedbu u vezi postavljenih kriterijuma i to smatram propustom ovog rada.
 \odgovor{Odgovoreno u sekciji \textit{Krupne primedbe i sugestije}.}
 
 \item Da li ima suštinskih grešaka i propusta?\\
 Ne, sve informacije priložene u radu su relevantne za zadatu temu. Iako ima nekih propusta, mislim da nije promašen smisao.
 
 \item Da li je naslov rada dobro izabran?\\
 Obzirom da je zadata tema ''Šta čini studije idealnim?'', smatram da se naslov rada može dodatno prilagoditi. Recimo: \textit{Kriterijumi idealnih studija}
 \odgovor{}
 
 \item Da li sažetak sadrži prave podatke o radu?\\
 Da, sadržaj taksativno prati obrađene celine i daje dobar uvod za čitaoca. Jedina zamerka je referisanje na github stranicu na kojoj se nalaze rezultati ankete (referisano je i u uvodu, gde je pravilnije da se navede).
 \odgovor{Ispravljeno, link je otklnonjen}
 
 \item Da li je rad lak-težak za čitanje?\\
 Zbog navedenih stilskih propusta, čitanje rad je u pojedinim sekcijama teže nego što bi trebalo da bude.
 \odgovor {Ispravljeno, već je navedeno u odeljku za sitne primedbe}
 
 \item Da li je za razumevanje teksta potrebno predznanje i u kolikoj meri?\\
 Za razumevanje teksta nije potrebno predznanje obzirom da su autori u uvodima sekcija objasnili sve što je obrađeno i nisu korišćeni termini i fraze za koje je potrebno imati dodatna stručna znanja.
 
 
 \item Da li je u radu navedena odgovarajuća literatura?\\
 U spisku literature fali barem jedna knjiga. Zamerka je i izmešana numeracija referenci (složiti reference tako da numeracija odgovara redosledu njihovog pojavljivanja u tekstu).
 \odgovor{Knjiga je postojala, ali je samo korišćeno značenje reči pošto je u pitanju bio rečnik. Promenjen je raspored referenci tako da bude odgovarajuć}
 
 \item Da li su u radu reference korektno navedene?\\
 Literatura je citirana i referisana na propisan način, kao i delovi teksta na koje se referiše u kasnijim sekcijama rada. Propust u referisanju napravljen je prilikom upotrebe slika i tabela - nisu sve tabele ni slike referisane u tekstu koji se oslanja na rezultate prikazane u tim elementima.
 \odgovor{Reference su ispravljene}
 
 \item Da li je struktura rada adekvatna?\\
 Struktura rada je u skladu sa pravilima. Rad sadrži sve potrebne sekcije i propisan sadržaj u njima. Pasusi su smisleni, obrađuju jedno po jednu temu i prelazi izmedju pasusa i sekcija su prirodni. U svakoj sekciji je objasnjeno šta će u okviru nje biti obradjeno, teme su najpre objasnjenje, a zatim su prikazani relevanti zaključci analize ankete i data su objasnjena zasto su važne i delimično su pokriveni načini za unapreživanje.
 
 \item Da li rad sadrži sve elemente propisane uslovom seminarskog rada (slike, tabele, broj strana...)?\\
 Da, rad je napisan na 12 strana čime nije probijena dozvoljena granica. Sadrži sve potrebne grafičke elemente dobijene prema zahtevu (originalne su i izvedene iz analiza koje su autori sproveli). Zadovoljeno je i donje ograničenje broja referenci u literaturi, ali ne i zahtev za vrstu literature (knjiga).
 \odgovor {Već je dat odgovor za zamerku za knjigu kao literaturu}
 
 \item Da li su slike i tabele funkcionalne i adekvatne?\\
 Slike i tabele daju adekvatne podatke, razumljive su i bez čitanja teksta, ali utvrđeno je da njihovo mesto u radu ne prati tekst na koji se odnose. Smatram da je takve slike, odnosno grafike potrebno premestiti. Takodje je primećeno da se pojedini brojčani rezultati navode u tekstu i ako postoje u tabeli, odnosna na grafiku, što nije po pravilima pisanja seminarskog rada. Naslovi tabela napisani su ispod tabela, čime nije ispunjeno pravilo obeležavanja - naslov je potrebno premesti iznad tabele.
 Forma priloženih ''pie charts''-ova ne zadovoljava uslov rasporeda delova po veličini, čime se gubi fokus pri analizi posmatranjem.
 
 \odgovor{Ispravljeno mesto slikama i tabelama. Naslovi tabela su ispravljeni}
 \odgovor{Što se tiče ponovnog navođenja brojčanih rezultata. Mislimo da je ova kritika pogrešna jer su slike tu kako bi dale vizuelnu reprezentaciju podataka i kako bi olakšale čitanje rada, a ne kao zamena za tekst. Ništa nije promenjeno}
 \odgovor{Nije nam poznato da dijagrami moraju imati navedenu formu. Smatramo da promena ne bi uticala na čitljivosti dijagrama. Ništa nije promenjeno}
 
 \end{enumerate}
 
 \section{Ocenite sebe}
 % Napišite koliko ste upućeni u oblast koju recenzirate: 
 % a) ekspert u datoj oblasti
 % b) veoma upućeni u oblast
 % c) srednje upućeni
 % d) malo upućeni 
 % e) skoro neupućeni
 % f) potpuno neupućeni
 % Obrazložite svoju odluku
 Smatram da je moj nivo upućenosti ''srednje upućen'' i da je zasnovan na ličnom iskustvu na Matematičkom fakultetu.
 
 
 \chapter{Recenzent \odgovor{--- ocena: 4} }
 
 
 \section{O čemu rad govori?}
 Rad analizira ključne kriterijume koje studenti smatraju važnim za ostvarenje idealnih studija, poput zadovoljstva kursevima, prilagođenosti tržištu rada i organizacije fakulteta. Na osnovu ankete sprovedene među studentima Matematičkog fakulteta, prikazani su rezultati koji ukazuju na izazove i mogućnosti za unapređenje kvaliteta studija.
 \section{Krupne primedbe i sugestije}
 U sekciji 4.1 navedeno je: "Naime, po mišljenjima autora, ocena je niža nego što bi trebalo da bude ako je cilj ostvariti idealne studije". Ovaj komentar je suvišan. Autor ne bi trebao da sugeriše subjektivni osećaj i da njime opovrgava rezultate istraživanja. Ukoliko autor ima primedbe na rezultate trebao bi da preispita reprezentativnost uzorka i da po potrebi ponovo sprovede anketu. Važno je naglasiti da ovo istraživanje ispituje pogled studenata a ne objektivno opisivanje idealnih studija.
 
 \odgovor{Navedena rečenica u 4.1 jeste pogrešna, ne bi trebalo da postoji u ovakvom tipu rada, otklonjena je.}
 
 \section{Sitne primedbe}
 Prvi pasus uvoda bi mogao biti napisan ozbiljnije i formalnije. U sekciji 4.4 iskorišćena fraza "...ukoliko pitate naše ispitanike"\ bi mogla biti promenjena u nešto ozbiljniju, npr. "...po rezultatima ankete". U zaključku postoji par gramatičkih grešaka (razmak posle zatvorene zagrade i pogrešno napisane reči "ima potencijala").
 
 \odgovor{Nije jasno na šta se misli kada se kaže ozbiljnije, svakako se slažemo da je prvi pasus uvoda delovao neozbiljno, ispravljen je. U sekciji 4.4 je ispravljena fraza, zvuči neformalno za ovaj vid rada. Ispravljene navedene greške u zaključku }
 
 
 \section{Provera sadržajnosti i forme seminarskog rada}
 Sadržaj je adekvatan i forma rada je ispunjena.
 \begin{enumerate}
 \item Da li rad dobro odgovara na zadatu temu?\\
 
 Da, rad se precizno bavi ključnim kriterijumima idealnih studija.
 \item Da li je nešto važno propušteno?\\
 
 Ne. Ključne stvari su obrađene i dobro analizirane.
 \item Da li ima suštinskih grešaka i propusta?\\
 Ne.
 
 \item Da li je naslov rada dobro izabran?\\
 Naslov "Idealne studije"\ nije dobro izabran jer ne naglašava da je cilj ispitati perspektivu studenata.
 \odgovor{Naslov rada je bio neprecizan, slažemo se, promenjen je u \emph{Ključne osobine idealnih studija} }
 
 \item Da li sažetak sadrži prave podatke o radu?\\
 Da, sažetak pruža dobar uvod u sadržaj rada.
 
 \item Da li je rad lak-težak za čitanje?\\
 Rad je lak i jasan za čitanje.
 
 \item Da li je za razumevanje teksta potrebno predznanje i u kolikoj meri?\\
 Da, potrebno je osnovno poznavanje akademske metodologije i studentskih programa.
 \odgovor{Nismo sigurni da li je ovo zamerka, ukoliko jeste, ne vidimo kako je moguće ispraviti je.}
 
 \item Da li je u radu navedena odgovarajuća literatura?\\
 Da, literatura je relevantna i adekvatno citirana.
 
 \item Da li su u radu reference korektno navedene?\\
 Da, reference su pravilno formatirane.
 
 \item Da li je struktura rada adekvatna?\\
 Da, struktura je adekvatna sa jasnim sekcijama.
 
 \item Da li rad sadrži sve elemente propisane uslovom seminarskog rada (slike, tabele, broj strana...)?\\
 Da.
 
 \item Da li su slike i tabele funkcionalne i adekvatne?\\
 Da. Slike i tabele su funkcionalne i adekvatne.
 
 \end{enumerate}
 
 \section{Ocenite sebe}
 Pokušao sam da apostrofiram ključne stvari koje se tiču metodologije i pristupa istraživanju. Pohvalio sam vrlo dobru analizu i strukturu rada ali i ostavio par jasnih komentara koji se tiču preciznosti i ispravnog formulisanja. Smatram da bi, pored ove, bilo dobro dobiti recenziju i od iskusnijeg stručnjaka kako bi se uvideli svi dobri i loši aspekti ovog rada. 
 \chapter{Recenzent \odgovor{--- ocena: 5} }
 
 
 \section{O čemu rad govori?}
 % Напишете један кратак пасус у којим ћете својим речима препричати суштину рада (и тиме показати да сте рад пажљиво прочитали и разумели). Обим од 200 до 400 карактера.
 Ovaj rad ispituje stavove studenata o kriterijumima koji čine studije 'idealnim' i analizira koliko su ti kriterijumi zastupljeni na Matematičkom fakultetu. Istraživanje je sprovedeno putem ankete, pri čemu su ispitane sledeće karakteristike: zadovoljstvo kursevima, priprema za samostalno usavršavanje, usklađenost kurseva sa tržištem, organizacija fakulteta i kurseva, kao i povratne informacije o napretku studenata. Doneti zaključci sugerišu moguća unapređenja koja se mogu sprovesti na Matematičkom fakultetu.
 
 
 
 \section{Krupne primedbe i sugestije}
 % Напишете своја запажања и конструктивне идеје шта у раду недостаје и шта би требало да се промени-измени-дода-одузме да би рад био квалитетнији.
 Struktura rada treba da bude drugačija. Osnovni zadaci koje je bilo potrebno ispuniti su:
     \begin{itemize}
         \item prikazati osnovne karakteristike idealnih studijskih programa i studija
         \item osmisliti i sprovesti anektu vezanu za studije na MatF-u koja proverava koliko tih kriterijuma koji su potrebni za idealne studije su ispunjeni na MatF-u
         \item prikazati rezultate ankete i obrazložiti zaključke koji se iz te ankete dobijaju 
     \end{itemize}
 Deo rada u kom studenti ocenjuju važnost ovih karakteristika predstavlja višak. On nije nužno beskoristan (možete ga iskoristiti za proveru da li su sve osobine za koje ste se odlučili važne), ali postoji prevelika sličnost između sadržaja tog podnaslova i narednog, u kom studenti izjašnjavaju svoje iskutvo. 
 \odgovor{Deo u kome se ocenjuje važnost karakteristika postoji kako bi se uvideo značaj zastupljenosti istih. Ako za neki kriterijum studenti smatraju da je nebitan, zašto onda diskutovati o zasupljenosti tog kriterijuma na fakultetu. Postoji dovoljno razlike između ovih navedenih naslova.}
 
 Predlažem da podnaslov 'Stavovi studenata' zamenite sa nekim u kom uvodite ključne karakteristike i dajete razloge zašto ste upravo njih izabrali. Ova izmena će doprineti čitljivosti, jer nećete imati toliko ponavljanja.
 \odgovor{Slažemo se, sam naslov je doprinosio nejasnoći rada. Zamenjen je.}
 \\
 \\
 Izbegavati duge i nejasne rečenice, kao i neformalne izraze. 
 \odgovor{Slažemo se, rečenice su preformulisane.}
 
 
 
 \begin{itemize}
     \item Primer iz uvoda: S obzirom da biti najbolji u bilo čemu nije nimalo lako, tako nije lako ni postaviti te kriterijume i jednoznačno ih odrediti. Da je lako, dosad bi verovatno sve studije bile idealne. 
     \item Može se zameniti sa: Definisanje uslova koji čine studije idealnim nije trivijalan zadatak.
 \end{itemize}
 
 
 \section{Sitne primedbe}
 % Напишете своја запажања на тему штампарских-стилских-језичких грешки
 
 \begin{itemize}
     \item Izbegavati korišćenje prvog lica množine (zaključujemo, sproveli smo...)
     
           Primer: Najpre smo sproveli anketu koja ispituje... zameniti sa: Sprovedena je anketa koja ispituje...
           
           Ne znam da li je ovo validna zamerka (u smislu da mora da se napravi ova izmena), ali mislim da je rešenje formalnije.
 
         \odgovor{Promenjeni su delovi teksta u kojima se javlja prvo lice množine. Slažemo se da je neformalno.}
           
     \item Podaci iz legende pojedinih grafova se ne slažu sa onim što graf predstavlja. 
 
           Slika 6 - prikazuje ocenjivanje organizacije Matematičkog fakulteta, dok u legendi piše slažem se itd.
         \odgovor{Promenjen opis slike da bude u skladu sa onim što je prikazano grafikonom.}
             
           Slika 7 - prikazuje koliko su često studenti dobijali povratne informacije o napretku, a ne njihove stavove o važnosti povratnih informacija. 
         \odgovor{Slažemo se sa primedbom, slika je pogrešna u potpunosti, izbačena je.}
 
           Slika 8.b - Ne odgovara na pitanje koliko su studenti samouvereni u svoje znanje.
         \odgovor{Slažemo se da je pogrešan opis slike, stoga je ispravaljen da bude u skladu sa grafikonom.}
         
 \end{itemize}
 
 
 \section{Provera sadržajnosti i forme seminarskog rada}
 % Oдговорите на следећа питања --- уз сваки одговор дати и образложење
 
 \begin{enumerate}
 \item Da li rad dobro odgovara na zadatu temu?\\
 Rad dobro odgovara na temu. Sve pomenute sugestije doprinose čitljivosti i preglednosti rada.
 \odgovor{Već je odgovoreno na ovo}
 
 \item Da li je nešto važno propušteno?\\
 Idealne studije treba da obezbede sve neophodne resurse i opremu za napredak studenata. Zaposleni profesori treba da budu kvalifikovani i dobri predavači. Školarina treba da bude tolerantna.
 \odgovor{Nemoguće je sprovesti kratku anketu, a da pritom pokrijemo sva pitanja. Zbog toga je postojalo pitanje otvorenog tipa da se može dodati komentar ukoliko nešto izostaje. Ove istaknute teme su samo lični stavovi recezenta}
 
 \item Da li ima suštinskih grešaka i propusta?\\
 Podnaslov u kome studenti ocenjuju karakteristike idealnih studija je suvišan.
 \odgovor{Nejasno na koji se podnaslov misli, ništa nije promenjeno.}
 
 
 \item Da li je naslov rada dobro izabran?\\
 Ne, naslov rada treba biti konkretniji. Na primer: 'Ključne osobine idealnih studija'
 \odgovor{Naslov rada jeste previše uopšten, promenjen je.}
 
 \item Da li sažetak sadrži prave podatke o radu?\\
 Prva rečenica u sažetku nije tačna, to nije cilj rada. Link ka rezultatima ankete ne treba da stoji u sažetku. Ostalo je u redu.
 \odgovor{Nije jasno na šta se misli pod tim da prva rečenica nije tačna. Svakako je promenjena tako da bude više u skladu sa opisima zahteva teme. Link je izbačen.}
 
 \item Da li je rad lak-težak za čitanje?\\
 Rad je težak za čitanje. Date su odgovarajuće sugestije.
 
 
 \item Da li je za razumevanje teksta potrebno predznanje i u kolikoj meri?\\
 Ne, tema rada je bliska svima koji studiraju ili onima koji su studije završili.
 
 
 \item Da li je u radu navedena odgovarajuća literatura?\\
 Nedostaje bar jedna adekvatna veb adresa sa linkom.
 \odgovor{Greškom je link izostao, ispravljeno.}
 
 \item Da li su u radu reference korektno navedene?\\
 Da, reference su pregledno i korektno navedene.
 
 
 \item Da li je struktura rada adekvatna?\\
 Ne, date su odgovarajuće sugestije.
 \odgovor{Već je odgovoreno na ovo}
 
 \item Da li rad sadrži sve elemente propisane uslovom seminarskog rada (slike, tabele, broj strana...)?\\
 Da, slike i tabele su pregledne i dobro odabrane. One imaju veliki doprinos u kvalitetu ovog rada.
 
 
 \item Da li su slike i tabele funkcionalne i adekvatne?\\
 Da, olakšavaju pregled podataka koje prikazuju.
 
 \end{enumerate}
 
 \section{Ocenite sebe}
 % Napišite koliko ste upućeni u oblast koju recenzirate: 
 % a) ekspert u datoj oblasti
 % b) veoma upućeni u oblast
 % c) srednje upućeni
 % d) malo upućeni 
 % e) skoro neupućeni
 % f) potpuno neupućeni
 % Obrazložite svoju odluku
 Kao neko ko je završio osnovne studije, smatram sebe veoma upućenim u oblast o kojoj je napisan rad.
 
 
 \chapter{Dodatne izmene}
 %Ovde navedite ukoliko ima izmena koje ste uradili a koje vam recenzenti nisu tražili. 
 
 \end{document}
 